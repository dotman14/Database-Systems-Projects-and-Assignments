\documentclass[12pt]{article}

\usepackage[margin=0.6in]{geometry}
\setlength{\parindent}{0em}
\usepackage{listings}
\usepackage{color}

\definecolor{dkgreen}{rgb}{0,0.6,0}
\definecolor{gray}{rgb}{0.5,0.5,0.5}
\definecolor{mauve}{rgb}{0.58,0,0.82}

\lstset{frame=tb,
	language=SQL,
	aboveskip=6mm,
	belowskip=6mm,
	showstringspaces=false,
	columns=flexible,
	basicstyle={\small\ttfamily},
	numbers=none, 
	numberstyle=\tiny\color{gray},
	keywordstyle=\color{blue},
	commentstyle=\color{dkgreen},
	numbersep=5pt
}

\author{
	\textbf{Oyedotun Oyesanmi } 
	\and 
	\textbf{Ming-Zhen Ling}}
\title{Database Management System - Project\\Phase 3}
\begin{document}
	\maketitle
	\pagebreak
	
	
\begin{lstlisting}
-- Oyedotun Oyesanmi and Ming-Zhen Liang
--Query 1
WITH sel(p_id) 
AS 
(
   SELECT "User".ID 
   FROM "User" 
   WHERE FIRST_NAME = 'Jim' 
   AND LAST_NAME = 'Champ'
),
les(i, ct) 
AS 
(
   SELECT TRC.TUTOR_USER_ID, count(DISTINCT TRC.COURSE_CODE || TRC.COURSE_NUMBER)
   FROM TUTOR_RESPONSIBLE_COURSES TRC 
   GROUP BY TRC.TUTOR_USER_ID
)
SELECT i 
FROM les 
JOIN sel 
ON i != p_id
WHERE
(
SELECT count(DISTINCT COURSE_CODE || COURSE_NUMBER)
FROM TUTOR_RESPONSIBLE_COURSES
WHERE TUTOR_USER_ID IN (p_id, i)) =
(
    SELECT min(ct) 
    FROM les 
    WHERE les.i IN (p_id)
)
	\end{lstlisting}
	
	
\begin{lstlisting}
--Query 2
SELECT UNIVERSITY.NAME, count(ACADEMIC_UNIT.UNIT_CAMPUS_ID)
FROM UNIVERSITY
INNER JOIN CAMPUS
   ON UNIVERSITY.ID = CAMPUS.UNIVERSITY_ID
INNER JOIN ACADEMIC_UNIT
   ON CAMPUS.ID = ACADEMIC_UNIT.UNIT_CAMPUS_ID
GROUP BY UNIVERSITY.NAME
\end{lstlisting}

\pagebreak

\begin{lstlisting}
--Query 3
SELECT UNIVERSITY.NAME, count(ACADEMIC_UNIT.UNIT_CAMPUS_ID)
FROM UNIVERSITY
LEFT OUTER JOIN CAMPUS
   ON UNIVERSITY.ID = CAMPUS.UNIVERSITY_ID
LEFT OUTER JOIN ACADEMIC_UNIT
   ON CAMPUS.ID = ACADEMIC_UNIT.UNIT_CAMPUS_ID
GROUP BY UNIVERSITY.NAME
\end{lstlisting}


\begin{lstlisting}
--Query 4
SELECT "User".FIRST_NAME, "User".LAST_NAME
FROM "User"
INNER JOIN FACULTY
   ON "User".ID = FACULTY.USER_ID
INNER JOIN STUDENT
   ON FACULTY.USER_ID = STUDENT.USER_ID
\end{lstlisting}


\begin{lstlisting}
--Query 5
SELECT "User".FIRST_NAME, "User".LAST_NAME
FROM "User"
WHERE ID IN
(
  SELECT TUTOR.USER_ID
  FROM TUTOR
  WHERE TUTOR.UNIT_COLLEGE_ID =
  (
    SELECT COLLEGE.UNIT_CLLGE_ID
	   FROM COLLEGE
	WHERE COLLEGE.ID = 'CLAS'
  )
)
\end{lstlisting}

\pagebreak

\begin{lstlisting}
--Query 6

CREATE VIEW R1
AS
(
	SELECT UNIT.UNIT_CLLGE_ID
	FROM UNIT
	WHERE UNIT.CAMPUS_ID = 'IUSB'
	AND UNIT.ID = 'CS'
);

CREATE VIEW T1
AS
(
	SELECT sum(ONCALL.HOURS) AS Hours, ONCALL.TUTOR_USER_ID AS Tut_ID, ONCALL.ID AS ID
	FROM ONCALL
	NATURAL JOIN R1
	WHERE ONCALL.ONCALLDATE >= TO_DATE('05-01-2005', 'MM/DD/YY')
	AND ONCALL.ONCALLDATE <= TO_DATE('05-30-2005', 'MM/DD/YY')
	AND ONCALL.STATUS = '3'
	AND ONCALL.TUTOR_USER_ID = 9003
	GROUP BY ONCALL.TUTOR_USER_ID, ONCALL.ID
)

CREATE VIEW T2
AS
(
	SELECT sum(HOURS) AS Hours, TUTOR_USER_ID AS Tut_ID, ID AS HiD
	FROM APPT
	NATURAL JOIN R1
	WHERE APPTDATE >= TO_DATE('05-01-2005', 'MM/DD/YY')
	AND APPTDATE <= TO_DATE('05-30-2005', 'MM/DD/YY')
	AND STATUS = '3'
	AND TUTOR_USER_ID = 9003
	GROUP BY TUTOR_USER_ID, ID
)

CREATE VIEW R231
AS
(
	SELECT TUTSESSION.UNIT_CLLGE_ID, TUTSESSION.ID, TUTSESSION.HOURS, TUTSESSION.TUTSESSIONDATE
	FROM R1
	JOIN TUTSESSION
	ON R1.UNIT_CLLGE_ID = TUTSESSION.UNIT_CLLGE_ID
)

CREATE VIEW R232
AS
(
	SELECT R231.UNIT_CLLGE_ID, R231.ID, R231.HOURS, R231.TUTSESSIONDATE
	FROM R231
	WHERE R231.TUTSESSIONDATE >= TO_DATE('05-01-2005', 'MM/DD/YY')
	AND R231.TUTSESSIONDATE <= TO_DATE('05-30-2005', 'MM/DD/YY')
)

CREATE VIEW R233
AS
(
	SELECT *
	FROM R232
	JOIN TUTOR_ASSIGNED_TUTSESSION
	ON R232.UNIT_CLLGE_ID = TUTOR_ASSIGNED_TUTSESSION.SESSION_UNIT_CLLGE_ID
	AND R232.ID = TUTOR_ASSIGNED_TUTSESSION.SESSION_ID
)

CREATE VIEW T3
AS
(
	SELECT sum(HOURS) AS Hours, TUTOR_USER_ID AS Tutor_ID, ID
	FROM R233
	WHERE R233.STATUS = '3'
	AND R233.TUTOR_USER_ID = 9003
	GROUP BY TUTOR_USER_ID, ID
)

SELECT 
T1.Hours + T2.Hours + T3.Hours
AS Total
FROM T1, T2, T3
\end{lstlisting}

\end{document}